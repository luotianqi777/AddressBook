% 详细设计
\chapter{详细设计}
\section{开发环境和工具}
	\begin{tabular}{|c|c|}
		\hline
		开发系统 & Linux(Ubuntu) \\
		\hline
		开发语言 & C++ \\
		\hline
		开发框架 & Qt \\
		\hline
		版本管理 & Git \\
		\hline
		数据库 & Sqlite \\
		\hline
		文档撰写 & \LaTeX \\
		\hline
	\end{tabular}
\section{开发规范}
	\begin{tabular}{|c|p{3cm}|}
		\hline
		规范对象 & 遵从的规范 \\
		\hline 
		变量命名规范 & 驼峰命名法 \\
		\hline
		数据库命名规范 & 由[a-z]及\_组成并尽可能言简意赅 \\
		\hline
		编码风格规范 & Google C++ Style \\
		\hline
	\end{tabular}
\section{窗口模块设计}
	\subsection{窗口移动模块}
		\begin{itemize}
			\item 功能\\鼠标拖拽窗口移动
			\item 程序逻辑\\
			\begin{tikzpicture}[node distance = 1.5cm]
				\node[startstop](start){程序启动} ;
				\node[process, below of = start](run){运行中} ;
				\node[decision, below of = run](click?){鼠标按下?} ;
				\node[left of = click?, xshift = -1cm](pt-1){} ;
				\coordinate(pt1) at (pt-1);
				\node[process, below of=click?](save){保存鼠标位置$pos_1$};
				\node[io, below of=save](move){鼠标移动,位置$pos_2$};
				\node[process,below of=move](cal){计算窗口相对移动位置$pos=pos_2-pos_1$};
				\node[right of=cal,xshift=3cm](pt-2){};
				\coordinate (pt2) at (pt-2);
				
				\draw [arrow] (start) -- (run) ;
				\draw [arrow] (run) -- (click?) ;
				\draw (click?) --node[above]{N} (pt1) ;
				\draw [arrow] (pt1) |- (run) ;
				\draw[arrow] (click?) --node[right]{Y} (save);
				\draw[arrow] (save) -- (move);
				\draw[arrow] (move) -- (cal);
				\draw (cal) -- (pt2);
				\draw[arrow] (pt2) |- (run);
			\end{tikzpicture}
		\end{itemize}
	\subsection{数据可编辑性设置模块}
		\begin{itemize}
			\item 功能 \\ 改变数据可编辑状态
			\item 程序逻辑 \\
			\begin{tikzpicture}[node distance=1.5cm]
				\node[startstop](start){开始};
				\node[process,below of=start](run){改变界面数据可编辑性};
				\node[startstop,below of=run](end){结束};
				\draw[arrow](start)--(run);
				\draw[arrow](run)--(end);
			\end{tikzpicture}
			\item 接口 
				\subitem void setEditable(bool)
		\end{itemize}
	\subsection{数据合法性检查模块}
		\begin{itemize}
			\item 功能 \\ 检查数据是否合法
			\item 程序逻辑 \\
			\begin{tikzpicture}[node distance=2cm]
				\node[startstop](start){开始};
				\node[decision,below of=start](ok?){是否合法?};
				\node[startstop,below of=ok?](true){返回True};
				\node[startstop,right of=ok?,xshift=1.5cm](false){返回False};
				\draw[arrow](start)--(ok?);
				\draw[arrow](ok?)--node[right]{Y}(true);
				\draw[arrow](ok?)--node[above]{N}(false);
			\end{tikzpicture}
			\item 接口 
				\subitem bool checkRight()
		\end{itemize}
	\subsection{数据显示模块}
		\begin{itemize}
			\item 功能\\将数据显示到界面上
			\item 程序逻辑\\
			\begin{tikzpicture}[node distance=2cm]
				\node[startstop](start){开始};
				\node[process,below of=start](show){将数据显示到界面上};
				\node[startstop,below of=show](end){结束};
				\draw[arrow](start)--(show);
				\draw[arrow](show)--(end);
			\end{tikzpicture}
			\item 接口
				\subitem void setInformation(Information)
		\end{itemize}
\section{数据库模块设计}
	\subsection{数据添加模块}
	\begin{itemize}
		\item 功能 \\ 向数据库中添加一条记录
		\item 程序逻辑\\
		\begin{tikzpicture}[node distance=2cm]
			\node[startstop](start){开始};
			\node[process,below of=start](run){写入数据};
			\node[decision,below of=run](ok?){写入是否成功?};
			\node[startstop,below of=ok?](true){返回True};
			\node[startstop,right of=ok?,xshift=2cm](false){返回False};
			\draw[arrow](start)--(run);
			\draw[arrow](run)--(ok?);
			\draw[arrow](ok?)--node[right]{Y}(true);
			\draw[arrow](ok?)--node[above]{N}(false);
		\end{tikzpicture}
		\item 接口 
			\subitem bool Append(int, Information)
	\end{itemize}
	\subsection{数据修改模块}
	\begin{itemize}
		\item 功能 \\ 修改数据库中一条记录
		\item 程序逻辑 \\
		\begin{tikzpicture}[node distance=2cm]
			\node[startstop](start){开始};
			\node[process,below of=start](run){修改数据};
			\node[decision,below of=run](ok?){修改是否成功?};
			\node[startstop,below of=ok?](true){返回True};
			\node[startstop,right of=ok?,xshift=2cm](false){返回False};
			\draw[arrow](start)--(run);
			\draw[arrow](run)--(ok?);
			\draw[arrow](ok?)--node[right]{Y}(true);
			\draw[arrow](ok?)--node[above]{N}(false);
		\end{tikzpicture}
		\item 接口 
			\subitem bool Change(int, Information)
	\end{itemize}
	\subsection{数据清空模块}
	\begin{itemize}
		\item 功能 \\ 删除所有数据
		\item 程序逻辑 \\
			\begin{tikzpicture}[node distance=2cm]
				\node[startstop](start){开始};
				\node[process,below of=start](run){删除数据};
				\node[decision,below of=run](ok?){删除是否成功?};
				\node[startstop,below of=ok?](true){返回True};
				\node[startstop,right of=ok?,xshift=2cm](false){返回False};
				\draw[arrow](start)--(run);
				\draw[arrow](run)--(ok?);
				\draw[arrow](ok?)--node[right]{Y}(true);
				\draw[arrow](ok?)--node[above]{N}(false);
			\end{tikzpicture}
		\item 接口 
			\subitem void DeleteAllDatas()
	\end{itemize}
	\subsection{数据获取模块}
	\begin{itemize}
		\item 功能 \\ 获取所有数据
		\item 程序逻辑 \\
		\begin{tikzpicture}[node distance=2cm]
			\node[startstop](start){开始};
			\node[process,below of=start](search){查询数据};
			\node[startstop,below of=search](end){返回查询结果};
			\draw[arrow](start)--(search);
			\draw[arrow](search)--(end);
		\end{tikzpicture}
		\item 接口 
			\subitem List<Information> GetAllDatas()
	\end{itemize}
\section{控制模块设计}
	\subsection{初始化模块}
	\begin{itemize}
		\item 功能\\从数据库读取数据初始化程序
		\item 程序逻辑\\
		\begin{tikzpicture}[node distance=2cm]
			\node[startstop](start){开始};
			\node[process,right of=start,xshift=2cm](get){从数据库读取所有数据};
			\node[process,below of=get](show){显示第一条数据};
			\node[process,left of=show,xshift=-2cm](editF){设置不可编辑};
			\node[startstop,below of=editF](end){结束};
			\draw[arrow](start)--(get);
			\draw[arrow](get)--(show);
			\draw[arrow](show)--(editF);
			\draw[arrow](editF)--(end);
		\end{tikzpicture}
		\item 接口
			\subitem void init()
	\end{itemize}
	\subsection{退出模块}
	\begin{itemize}
		\item 功能\\储存数据并退出程序
		\item 程序逻辑\\
		\begin{tikzpicture}[node distance=2cm]
			\node[startstop](start){开始};
			\node[process,below of=start](delete){删除数据库所有数据};
			\node[process,below of=delete](save){将当前所有数据添加到数据库};
			\node[startstop,below of=save](end){退出程序};
			\draw[arrow](start)--(delete);
			\draw[arrow](delete)--(save);
			\draw[arrow](save)--(end);
		\end{tikzpicture}
		\item 接口 on\_pushButtonExit\_clicked()
	\end{itemize}
	\subsection{记录添加模块}
	\begin{itemize}
		\item 功能\\添加一条记录
		\item 程序逻辑\\
		\begin{tikzpicture}[node distance=2cm]
			\node[startstop](start){开始};
			\node[process,below of=start](editT){设置可编辑};
			\node[io,below of=editT](input){编辑信息};
			\node[process,below of=input](modify){进入记录修改模块};
			\draw[arrow](start)--(editT);
			\draw[arrow](editT)--(input);
			\draw[arrow](input)--(modify);
		\end{tikzpicture}
		\item 接口
			\subitem void on\_pushButtonAdd\_clicked()
	\end{itemize}
	\subsection{记录删除模块}
	\begin{itemize}
		\item 功能 \\删除记录
		\item 程序逻辑\\
		\begin{tikzpicture}[node distance=2cm]
			\node[startstop](start){开始};
			\node[process,below of=start](delete){删除信息};
			\node[process,below of=delete](set){设置新的信息};
			\node[startstop,below of=set](end){结束};
			\draw[arrow](start)--(delete);
			\draw[arrow](delete)--(set);
			\draw[arrow](set)--(end);
		\end{tikzpicture}
		\item 接口
			\subitem void on\_pushButtonDelete\_clicked()
	\end{itemize}
	\subsection{记录修改模块}
	\begin{itemize}
		\item 功能\\修改记录
		\item 程序逻辑\\
		\begin{tikzpicture}[node distance=2cm]
			\node[startstop](start){开始};
			\node[io,below of=start](input){输入};
			\node[decision,below of=input](edit?){窗口是否在可编辑状态};
			\node[decision,below of=edit?,right of=edit?](ok?){输入是否合法};
			\node[right of=ok?,xshift=1cm](pt-1){};
			\coordinate(pt1) at (pt-1);
			\node[process,below of=edit?,left of=edit?](editF){设置不可编辑状态};
			\node[process,below of=ok?](editT){设置可编辑状态};
			\node[startstop,below of=editF](end){结束};
			\draw[arrow](start)--(input);
			\draw[arrow](input)--(edit?);
			\draw[arrow](edit?)--node[left]{可编辑}(editF);
			\draw[arrow](edit?)--node[right]{不可编辑}(ok?);
			\draw[arrow](ok?)--node[right]{Y}(editT);
			\draw(ok?)--node[above]{N}(pt1);
			\draw[arrow](pt1)|-(input);
			\draw[arrow](editF)--(end);
			\draw[arrow](editT)--(end);
		\end{tikzpicture}
		\item 接口
			\subitem void on\_pushButtonModify\_clicked()
	\end{itemize}
	\subsection{联系人更改模块}
	\begin{itemize}
		\item 功能\\更变联系人时更改显示信息
		\item 程序逻辑\\
		\begin{tikzpicture}[node distance=2cm]
			\node[startstop](start){开始};
			\node[process,below of=start](get){获取选中联系人编号$id$};
			\node[process,below of=get](set){设置新的数据};
			\node[startstop,below of=set](end){结束};
			\draw[arrow](start)--(get);
			\draw[arrow](get)--(set);
			\draw[arrow](set)--(end);
		\end{tikzpicture}
		\item 接口
			\subitem void on\_listWidgetAddress\_itemSelectionChanged()
	\end{itemize}