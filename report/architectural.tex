% 概要设计
\chapter{概要设计}
\section{总体模块设计}
	\begin{tabular}{|c|l|l|}
		\hline
		母模块名称& 子模块名称 & 模块功能 \\
		\hline 
		\multirow{4}{*}{ 窗口模块}
		& 窗口移动模块 & 鼠标拖动窗口移动 \\
		\cline{2-3}
		& 数据可编辑性设置模块 & 改变数据可编辑性状态 \\
		\cline{2-3}
		& 数据合法检查模块 & 检查数据是否合法 \\
		\cline{2-3}
		& 数据显示模块 & 将数据显示到界面上 \\
		\hline
		\multirow{4}{*}{
		数据库模块} 
		& 数据添加模块 & 向数据库添加数据 \\
		\cline{2-3}
		& 数据修改模块 & 从数据库中修改数据 \\
		\cline{2-3}
		& 数据获取模块 & 读取数据库的数据 \\
		\cline{2-3}
		& 数据清空模块 & 删除数据库的所有数据 \\
		\hline
		\multirow{5}{*}{控制模块}
		& 初始化模块 & 程序启动时从数据库中读取数据用于初始化界面 \\
		\cline{2-3}
		& 退出模块 & 程序退出时将数据同步到数据库 \\
		\cline{2-3}
		& 记录添加模块 & 向UI中添加记录 \\
		\cline{2-3}
		& 记录删除模块 & 从UI中删除记录 \\
		\cline{2-3}
		& 记录修改模块 & 从UI中修改记录 \\
		\hline
	\end{tabular}
\section{数据结构设计}
	\subsection{逻辑结构设计}
		\begin{itemize}
			\item[-] Class Name: Information
			\item ID(索引)
			\item 姓名
			\item 性别
			\item 移动电话(唯一)
			\item 固定电话
			\item 备注
			\item 其他
		\end{itemize}
	\subsection{物理结构设计}
		\begin{tabular}{|l|l|l|}
			\hline
			\multicolumn{3}{|c|}{Class Name: Information} \\
			\hline
			修饰符 & 类型 & 标识符 \\
			\hline
			private & int & gender \\
			private & int & fixNumber \\
			private & int & mobileNumber \\
			private & string & name \\
			private & string & remark \\
			private & string & otherInfo \\
			\hline
			public & int & getGender() \\
			public & int & getFixNumber() \\
			public & int & getMobileNumber() \\
			public & string & getName() \\
			public & string & getRemark() \\
			public & string & getOtherInfo() \\
			public & void & setGender(int) \\
			public & void & setFixNumber(int) \\
			public & void & setMobileNumber(int) \\
			public & void & setName(string) \\
			public & void & setRemark(string) \\
			public & void & setOtherInfo(string) \\
			\hline
		\end{tabular}
	\subsection{数据结构与程序的关系}
		\begin{itemize}
			\item 用于储存从数据库读取的数据并通过控制模块显示到UI上。
			\item 用于储存从UI上读取的数据并通过控制模块同步到数据库。
	\end{itemize}
\section{接口设计}
		\begin{tabular}{|l|l|l|p{3cm}|}
			\hline
			所属模块 &返回类型 & 名称  & 功能 \\
			\hline
			\multirow{3}{*}{窗口模块}
			&bool & checkRight() &  \\
			&void & setEditable(bool)  & 设置可编辑性 \\
			&void & setInformation(Information) & 设置UI界面信息 \\
			\hline			
			\multirow{6}{*}{控制模块}
			&void & init() &  初始化 \\
			&void & on\_pushButtonExit\_clicked() &  点击关闭按钮时储存数据并退出程序 \\
			&void & on\_pushButtonDelete\_clicked() &  点击删除按钮时删除选中记录 \\
			&void & on\_pushButtonModify\_clicked() &  点击修改按钮时开始修改记录 再次点击时完成修改 \\
			&void & on\_pushButtonAdd\_clicked() &  点击添加按钮时添加一条记录 \\
			&void & on\_listWidgetAddress\_itemSelectionChanged() & 联系人选中项变更时更换显示数据 \\
			\hline
			\multirow{4}{*}{数据库模块}
			&bool & Append(int id,Information data) &  添加一条数据 \\
			&bool & Change(int id, Information data) & 更改一条数据 \\
			&bool & DeleteAllData() &  删除所有数据 \\
			&List<Information> & GetAllDatas() &  获取所有数据 \\
			\hline
		\end{tabular}
\section{运行设计}
	\subsection{运行模块的组合}
		\begin{tikzpicture}[node distance = 2cm]
		\node (control) [startstop] {控制模块} ;
		\node (ui) [startstop, left of = control, xshift = -1cm, yshift = -2cm] {窗口模块} ;
		\node (database) [startstop, right of = control, xshift = 1cm, yshift = -2cm] {数据库模块} ;
		\draw (ui) -- (control) ;
		\draw (database) -- (control) ;
		\end{tikzpicture}
	\subsection{运行控制流程}	
		\begin{tikzpicture}[node distance = 2cm]
		\node (init) [process] {界面初始化} ;
		\node (data) [process, right of = init, xshift = 3cm] {窗口模块} ;
		\node (start) [startstop, right of = data, xshift = 2cm] {启动程序} ;
		\node (input) [io, below of = init] {输入} ;
		\node (control) [process, below of = input] {控制模块} ;
		\node (exit?) [decision, right of = control, xshift = 3cm] {退出程序?} ;
		\node (run) [process, below of = exit?, yshift = 4cm] {处理输入} ;
		\node (end) [startstop, right of = exit?, xshift = 2cm] {退出程序} ;
		
		
		\draw [arrow] (start) -- (data) ;
		\draw [arrow] (data) --node[above]{控制模块}node[below]{数据库模块} (init) ;
		\draw [arrow] (init) -- (input) ;
		\draw [arrow] (input) -- (control) ;
		\draw [arrow] (control) -- (exit?) ;
		\draw [arrow] (exit?) -- node[above]{Y} (end) ;
		\draw [arrow] (exit?) -- node[right]{N} (run) ;
		\draw [arrow] (run) -- (input) ;
		\end{tikzpicture}
