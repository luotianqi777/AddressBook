\documentclass[answer]{exam}
\usepackage{ctex}
\usepackage{underlin}
\usepackage[left=1.25in,right=1.25in,top=1in,bottom=1in]{geometry}

\title{软件工程复习题}
\author{}
\date{}

\begin{document}
\maketitle
\section{单选题}
\begin{questions}
	\question 软件开发的结构化设计(SD)方法,权标知道模块划分的最重要原则应该是()。\\
	\begin{oneparchoices}
		\choice 模块高内聚
		\choice 模块高耦合
		\choice 模块独立性
		\choice 程序模块化
	\end{oneparchoices}
	\question 软件工程方法的提出起源于软件危机,而其目的应该是最终解决软件的什么问题?()\\
	\begin{oneparchoices}
		\choice 产生危机
		\choice 质量保证
		\choice 开发效率
		\choice 生产工程化
	\end{oneparchoices}
	\question 软件工程开发的可行性研究是决定软件项目是否继续开发的关键,而可行性研究的结论主要相关于()\\
	\begin{oneparchoices}
		\choice 软件系统目标
		\choice 软件的性能
		\choice 软件的功能
		\choice 软件的质量
	\end{oneparchoices}
	\question 软件需求分析一般应确定的时用户对软件的()\\
	\begin{oneparchoices}
		\choice 功能需求
		\choice 非功能需求
		\choice 性能需求
		\choice 功能需求和非功能需求
	\end{oneparchoices}
	\question 软件测试是满足软件的功能和性能要求,保证软件正确性的措施,一般软件测试计划的制定应始于软件开发的哪个阶段?()\\
	\begin{oneparchoices}
		\choice 需求分析
		\choice 软件分析
		\choice 程序编码
		\choice 软件计划
	\end{oneparchoices}
	\question 软件工程方法是在时间中不断发展的方法,而早起的团建工程方法主要是指?()\\
	\begin{oneparchoices}
		\choice 原型化方法
		\choice 结构化方法
		\choice 面向对象方法
		\choice 功能分解法
	\end{oneparchoices}
	\question 数据流图描述数据在软件中流动和被处理变换的过程,它是以图示的方法来表示,即()\\
	\begin{oneparchoices}
		\choice 软件模型
		\choice 软件功能
		\choice 软件结构
		\choice 软件加工
	\end{oneparchoices}
	\question 软件工程学涉及到软件开发技术和工程管理两方面的内容,下述内容中哪个不属于开发技术的范畴?()\\
	\begin{oneparchoices}
		\choice 软件开发方法
		\choice 软件开发工具
		\choice 软件工程环境
		\choice 软件工程经济
	\end{oneparchoices}
	\question 软件文档是软件工程实施中的重要成分,它不仅是软件开发的各阶段的重要依据,而且也影响软件的()\\
	\begin{oneparchoices}
		\choice 可理解性
		\choice 可维护性
		\choice 可扩展性
		\choice 可靠性
	\end{oneparchoices}
	\question 从()语言开始,软件摆脱了对硬件的依赖。\\
	\begin{oneparchoices}
		\choice 第一代
		\choice 第二代
		\choice 第三代
		\choice 第四代
	\end{oneparchoices}
	\question 在下面列出的基本成分中,哪个不是实体关系图的基本成分?()\\
	\begin{oneparchoices}
		\choice 实体
		\choice 数据存储
		\choice 关系
		\choice 属性
	\end{oneparchoices}
	\question 在下面的概念模式中,哪种描述的不是动态数据结构或属性?()\\
	\begin{oneparchoices}
		\choice 实体
		\choice 结构图
		\choice 实体关系图
		\choice 数据流程图
	\end{oneparchoices}
	\question 结构化程序设计主要强调程序的()\\
	\begin{oneparchoices}
		\choice 效率
		\choice 速度
		\choice 可读性
		\choice 大小
	\end{oneparchoices}
	\question 在软件测试中根据程序的功能说明,而不关心程序内部逻辑的测试方法为()\\
	\begin{oneparchoices}
		\choice 黑盒法
		\choice 白盒法
		\choice 灰盒法
		\choice 综合法
	\end{oneparchoices}
	\question 软件开发的结构化分析方法,常用的描述软件功能需求的工具有()\\
	\begin{oneparchoices}
		\choice 业务流程图,数据字典
		\choice 软件流程图,模块说明\\
		\choice 数据流图,数据字典
		\choice 系统流程图,程序编码
	\end{oneparchoices}
	\question 结构化程序设计思想的核心是要求程序只由顺序、循环和()三种结构组成。\\
	\begin{oneparchoices}
		\choice 分支
		\choice 单入口
		\choice 单出口
		\choice 有规则GOTO
	\end{oneparchoices}
	\question 软件生存周期中,所占时间最长的时哪个阶段()。\\
	\begin{oneparchoices}
		\choice 分析与设计
		\choice 维护
		\choice 编码
		\choice 测试
	\end{oneparchoices}
	\question 确定软件系统的主要功能,即进行系统功能分析,提出软件系统的目标,范围与功能说明成为结构化方法中的()。\\
	\begin{oneparchoices}
		\choice 需求分析
		\choice 可行性分析
		\choice 总体设计
		\choice 问题定义
	\end{oneparchoices}
	\question 在结构化方法中,软件功能分解应属于软件开发中的哪一阶段?()\\
	\begin{oneparchoices}
		\choice 总体设计
		\choice 需求分析
		\choice 详细设计
		\choice 编程调试
	\end{oneparchoices}
	\question 下列哪一种软件设计方法是基于动态定义需求的设计方法?()\\
	\begin{oneparchoices}
		\choice 结构化分析方法(SA)
		\choice 面向对象的软件开发方法(SD)\\
		\choice 结构化设计方法
		\choice 原型化方法
	\end{oneparchoices}
	\question 在软件结构化设计中,好的软件结构设计应该力求做到()\\
	\begin{oneparchoices}
		\choice 顶层扇出较少,中间扇出较高,底层模块低扇入\\
		\choice 顶层扇出较高,中间扇出较少,底层模块高扇入\\
		\choice 顶层扇入较少,中间扇出较高,底层模块高扇入\\
		\choice 顶层扇入较高,中间扇入较少,底层模块低扇入
	\end{oneparchoices}
	\question 软件开发的结构生命周期法(SA)的基本假定是认为软件需求能做到()\\
	\begin{oneparchoices}
		\choice 严格定义
		\choice 初步定义
		\choice 早期冻结
		\choice 动态改变
	\end{oneparchoices}
	\question 软件工程学中除重视软件开发技术的研究外,另一重要组成内容是软件的()\\
	\begin{oneparchoices}
		\choice 工程管理
		\choice 成本核算
		\choice 人工培训
		\choice 工具开发
	\end{oneparchoices}
	\question 软件设计包括总体设计和详细设计两部分,下列陈述中哪个是详细设计的内容?()\\
	\begin{oneparchoices}
		\choice 软件结构
		\choice 数据库设计
		\choice 定制测试计划
		\choice 模块算法
	\end{oneparchoices}
	\question 软件开发的面向对象OO方法,常用的描述软件功能需求的工具是()\\
	\begin{oneparchoices}
		\choice 用例图
		\choice 类图
		\choice 活动图
		\choice 顺序图
	\end{oneparchoices}
	\question 在软件测试方法中,黑盒测试法和白盒测试法是常用的方法,其中黑盒测试法主要是用于测试()\\
	\begin{oneparchoices}
		\choice 结构合理性
		\choice 软件外部功能
		\choice 程序正确性
		\choice 程序内部逻辑
	\end{oneparchoices}
	\question 数据字典是软件需求分析阶段的最重要的工具之一,其最基本的功能是()\\
	\begin{oneparchoices}
		\choice 数据库设计
		\choice 数据通讯
		\choice 数据定义
		\choice 数据维护
	\end{oneparchoices}
	\question 软件测试是软件开发过程中重要和不可缺少的阶段,其包含的的内容和步骤甚多,而在测试过程的多种环节中最基本的是()\\
	\begin{oneparchoices}
		\choice 集成测试
		\choice 单元测试
		\choice 系统测试
		\choice 验收测试
	\end{oneparchoices}
	\question 软件工程开发的可行性研究是决定软件项目是否继续开发的关键,而可行性研究的结论主要关于()\\
	\begin{oneparchoices}
		\choice 软件系统目标
		\choice 软件的可测试性
		\choice 软件的功能
		\choice 软件的质量
	\end{oneparchoices}
	\question 结构化程序设计理论认为,实现良好的程序结构要应用()的分析方法。\\
	\begin{oneparchoices}
		\choice 自顶向下
		\choice 自底向上
		\choice 面向对象
		\choice 基于组件
	\end{oneparchoices}
	\question 在下面列出的基本成分中,哪个不是数据流程图的基本成分?()\\
	\begin{oneparchoices}
		\choice 信息处理
		\choice 信息储存
		\choice 外部实体
		\choice 系统状态
	\end{oneparchoices}
	\question PAD(\uline{Proble} \uline{Iiinlysis} \uline{Diagam})图是一种()工具。\\
	\begin{oneparchoices}
		\choice 系统描述
		\choice 详细设计
		\choice 测试
		\choice 编程辅助
	\end{oneparchoices}
	\question 程序设计属于软件开发过程的()阶段。\\
	\begin{oneparchoices}
		\choice 设计
		\choice 编程
		\choice 实现
		\choice 编码
	\end{oneparchoices}
	\question 程序的三种基本控制结构,它们的共同点是()\\
	\begin{oneparchoices}
		\choice 不能嵌套使用
		\choice 只能用来写简单的程序\\
		\choice 已经用硬件实现
		\choice 只有一个入口和一个出口
	\end{oneparchoices}
	\question 耦合是软件中各模块件互相联系的一种度量,耦合的强弱取决于模块间的复杂程度。耦合的若干种类中,耦合度最高的是()\\
	\begin{oneparchoices}
		\choice 内容耦合
		\choice 非直接耦合
		\choice 数据耦合
		\choice 控制耦合
	\end{oneparchoices}
	\question 在软件工程中,软件测试的目的是()\\
	\begin{oneparchoices}
		\choice 实验性运行软件
		\choice 发现软件错误\\
		\choice 证明软件是正确的
		\choice 找出软件中全部错误
	\end{oneparchoices}
	\question 下面哪一项不是软件设计规格说明中模块的内容?()\\
	\begin{oneparchoices}
		\choice 接口描述
		\choice 数据的组织
		\choice 外部文件结构
		\choice 处理过程描述
	\end{oneparchoices}
	\question 需求分析的主要任务是()\\
	\begin{oneparchoices}
		\choice 确定软件系统的主要功能,即进行系统功能分析,提出软件系统的目标、范围与功能说明\\
		\choice 分析用户要求,将软件功能和性能描述为具体的规格说明书\\
		\choice 对问题定义阶段所确定的问题实现的可能性和必要性做出研究\\
		\choice 建立软件系统的总体结构,子系统划分,并提出软件结构图
	\end{oneparchoices}
	\question 软件结构中,由一模块直接控制的其他模块数称为()\\
	\begin{oneparchoices}
		\choice 深度
		\choice 宽度
		\choice 扇入数
		\choice 扇出数
	\end{oneparchoices}
	\question 在数据字典中,()给出了某个文件的定义,文件的定义通常也是列出其记录的组成。\\
	\begin{oneparchoices}
		\choice 加工
		\choice 数据存储
		\choice 数据流
		\choice 数据项
	\end{oneparchoices}
	\question 面向数据流的软件设计方法,一般是把数据流图中的数据流划分为什么样的两种流,再将数据流图映射为软件结构?()\\
	\begin{oneparchoices}
		\choice 数据流与事务流
		\choice 交换流与事务流
		\choice 信息流与控制流
		\choice 交换流与数据流
	\end{oneparchoices}
	\question 与早期的软件开发方式相比较,结构化周期法最重要的指导原则应该是()\\
	\begin{oneparchoices}
		\choice 自顶向下设计
		\choice 分阶段开发
		\choice 逐步求精
		\choice 用户需求至少
	\end{oneparchoices}
	\question 软件计划是软件开发的早期和重要阶段,此阶段要求交互和配合的是()\\
	\begin{oneparchoices}
		\choice 设计人员和用户
		\choice 分析人员和用户\\
		\choice 分析人员和设计人员
		\choice 编码人员和用户
	\end{oneparchoices}
	\question 数据字典时对数据定义信息的集合,它所定义的对象都包含于()\\
	\begin{oneparchoices}
		\choice 数据流图
		\choice 程序框图
		\choice 软件结构
		\choice 方案图
	\end{oneparchoices}
	\question 在软件开发的各种资源中,()是最重要的资源。\\
	\begin{oneparchoices}
		\choice 开发工具
		\choice 方法
		\choice 硬件环境
		\choice 人员
	\end{oneparchoices}
	\question 软件文档是软件工程实施中的重要成分,它不仅是软件开发各阶段的重要依据,而且也影响软件的()\\
	\begin{oneparchoices}
		\choice 可理解性
		\choice 可维护性
		\choice 可扩展性
		\choice 可移植性
	\end{oneparchoices}
	\question 软件复审时,其主要的复审对象是()\\
	\begin{oneparchoices}
		\choice 软件结构
		\choice 软件文档
		\choice 系统编程
		\choice 文档标准
	\end{oneparchoices}
	\question 判断树和判断表是用于描述结构化分析方法中()环节的工具。\\
	\begin{oneparchoices}
		\choice 功能说明
		\choice 数据加工
		\choice 流程描述
		\choice 性能说明
	\end{oneparchoices}
	\question 在体系结构图这种概念模型中,矩形框表示()\\
	\begin{oneparchoices}
		\choice 处理过程
		\choice 模块
		\choice 外部实体
		\choice 内部实体
	\end{oneparchoices}
	\question 主要用来描述系统状态及其转换方式的数据模式是()\\
	\begin{oneparchoices}
		\choice E-R图
		\choice 结构图
		\choice DFD图
		\choice IPO图
	\end{oneparchoices}
\end{questions}
\end{document}