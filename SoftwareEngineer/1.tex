\documentclass[answers]{exam}
\usepackage{ctex}
\usepackage{underlin}
\usepackage[left=1.25in,right=1.25in,top=1in,bottom=1in]{geometry}

\title{软件工程复习题}
\author{}
\date{}

\begin{document}
\maketitle
\section{单选题}
\begin{questions}
	\question 软件开发的结构化设计(SD)方法,权标知道模块划分的最重要原则应该是()。\\
	\begin{oneparchoices}
		\choice 模块高内聚
		\choice 模块高耦合
		\correctchoice 模块独立性
		\choice 程序模块化
	\end{oneparchoices}
	\question 软件工程方法的提出起源于软件危机,而其目的应该是最终解决软件的什么问题?()\\
	\begin{oneparchoices}
		\choice 产生危机
		\choice 质量保证
		\choice 开发效率
		\correctchoice 生产工程化
	\end{oneparchoices}
	\question 软件工程开发的可行性研究是决定软件项目是否继续开发的关键,而可行性研究的结论主要相关于()\\
	\begin{oneparchoices}
		\correctchoice 软件系统目标
		\choice 软件的性能
		\choice 软件的功能
		\choice 软件的质量
	\end{oneparchoices}
	\question 软件需求分析一般应确定的是用户对软件的()\\
	\begin{oneparchoices}
		\choice 功能需求
		\choice 非功能需求
		\choice 性能需求
		\correctchoice 功能需求和非功能需求
	\end{oneparchoices}
	\question 软件测试是满足软件的功能和性能要求,保证软件正确性的措施,一般软件测试计划的制定应始于软件开发的哪个阶段?()\\
	\begin{oneparchoices}
		\correctchoice 需求分析
		\choice 软件分析
		\choice 程序编码
		\choice 软件计划
	\end{oneparchoices}
	\question 软件工程方法是在时间中不断发展的方法,而早起的团建工程方法主要是指?()\\
	\begin{oneparchoices}
		\choice 原型化方法
		\correctchoice 结构化方法
		\choice 面向对象方法
		\choice 功能分解法
	\end{oneparchoices}
	\question 数据流图描述数据在软件中流动和被处理变换的过程,它是以图示的方法来表示,即()\\
	\begin{oneparchoices}
		\choice 软件模型
		\correctchoice 软件功能
		\choice 软件结构
		\choice 软件加工
	\end{oneparchoices}
	\question 软件工程学涉及到软件开发技术和工程管理两方面的内容,下述内容中哪个不属于开发技术的范畴?()\\
	\begin{oneparchoices}
		\choice 软件开发方法
		\choice 软件开发工具
		\choice 软件工程环境
		\correctchoice 软件工程经济
	\end{oneparchoices}
	\question 软件文档是软件工程实施中的重要成分,它不仅是软件开发的各阶段的重要依据,而且也影响软件的()\\
	\begin{oneparchoices}
		\choice 可理解性
		\correctchoice 可维护性
		\choice 可扩展性
		\choice 可靠性
	\end{oneparchoices}
	\question 从()语言开始,软件摆脱了对硬件的依赖。\\
	\begin{oneparchoices}
		\choice 第一代
		\choice 第二代
		\correctchoice 第三代
		\choice 第四代
	\end{oneparchoices}
	\question 在下面列出的基本成分中,哪个不是实体关系图的基本成分?()\\
	\begin{oneparchoices}
		\choice 实体
		\correctchoice 数据存储
		\choice 关系
		\choice 属性
	\end{oneparchoices}
	\question 在下面的概念模式中,哪种描述的不是动态数据结构或属性?()\\
	\begin{oneparchoices}
		\choice 实体
		\choice 结构图
		\correctchoice 实体关系图
		\choice 数据流程图
	\end{oneparchoices}
	\question 结构化程序设计主要强调程序的()\\
	\begin{oneparchoices}
		\choice 效率
		\choice 速度
		\correctchoice 可读性
		\choice 大小
	\end{oneparchoices}
	\question 在软件测试中根据程序的功能说明,而不关心程序内部逻辑的测试方法为()\\
	\begin{oneparchoices}
		\correctchoice 黑盒法
		\choice 白盒法
		\choice 灰盒法
		\choice 综合法
	\end{oneparchoices}
	\question 软件开发的结构化分析方法,常用的描述软件功能需求的工具有()\\
	\begin{oneparchoices}
		\choice 业务流程图,数据字典
		\choice 软件流程图,模块说明\\
		\correctchoice 数据流图,数据字典
		\choice 系统流程图,程序编码
	\end{oneparchoices}
	\question 结构化程序设计思想的核心是要求程序只由顺序、循环和()三种结构组成。\\
	\begin{oneparchoices}
		\correctchoice 分支
		\choice 单入口
		\choice 单出口
		\choice 有规则GOTO
	\end{oneparchoices}
	\question 软件生存周期中,所占时间最长的是哪个阶段()。\\
	\begin{oneparchoices}
		\choice 分析与设计
		\correctchoice 维护
		\choice 编码
		\choice 测试
	\end{oneparchoices}
	\question 确定软件系统的主要功能,即进行系统功能分析,提出软件系统的目标,范围与功能说明成为结构化方法中的()。\\
	\begin{oneparchoices}
		\correctchoice 需求分析
		\choice 可行性分析
		\choice 总体设计
		\choice 问题定义
	\end{oneparchoices}
	\question 在结构化方法中,软件功能分解应属于软件开发中的哪一阶段?()\\
	\begin{oneparchoices}
		\correctchoice 总体设计
		\choice 需求分析
		\choice 详细设计
		\choice 编程调试
	\end{oneparchoices}
	\question 下列哪一种软件设计方法是基于动态定义需求的设计方法?()\\
	\begin{oneparchoices}
		\choice 结构化分析方法(SA)
		\correctchoice 面向对象的软件开发方法(SD)\\
		\choice 结构化设计方法
		\choice 原型化方法
	\end{oneparchoices}
	\question 在软件结构化设计中,好的软件结构设计应该力求做到()\\
	\begin{oneparchoices}
		\choice 顶层扇出较少,中间扇出较高,底层模块低扇入\\
		\correctchoice 顶层扇出较高,中间扇出较少,底层模块高扇入\\
		\choice 顶层扇入较少,中间扇出较高,底层模块高扇入\\
		\choice 顶层扇入较高,中间扇入较少,底层模块低扇入
	\end{oneparchoices}
	\question 软件开发的结构生命周期法(SA)的基本假定是认为软件需求能做到()\\
	\begin{oneparchoices}
		\correctchoice 严格定义
		\choice 初步定义
		\choice 早期冻结
		\choice 动态改变
	\end{oneparchoices}
	\question 软件工程学中除重视软件开发技术的研究外,另一重要组成内容是软件的()\\
	\begin{oneparchoices}
		\correctchoice 工程管理
		\choice 成本核算
		\choice 人工培训
		\choice 工具开发
	\end{oneparchoices}
	\question 软件设计包括总体设计和详细设计两部分,下列陈述中哪个是详细设计的内容?()\\
	\begin{oneparchoices}
		\choice 软件结构
		\choice 数据库设计
		\choice 定制测试计划
		\correctchoice 模块算法
	\end{oneparchoices}
	\question 软件开发的面向对象OO方法,常用的描述软件功能需求的工具是()\\
	\begin{oneparchoices}
		\correctchoice 用例图
		\choice 类图
		\choice 活动图
		\choice 顺序图
	\end{oneparchoices}
	\question 在软件测试方法中,黑盒测试法和白盒测试法是常用的方法,其中黑盒测试法主要是用于测试()\\
	\begin{oneparchoices}
		\choice 结构合理性
		\correctchoice 软件外部功能
		\choice 程序正确性
		\choice 程序内部逻辑
	\end{oneparchoices}
	\question 数据字典是软件需求分析阶段的最重要的工具之一,其最基本的功能是()\\
	\begin{oneparchoices}
		\choice 数据库设计
		\choice 数据通讯
		\correctchoice 数据定义
		\choice 数据维护
	\end{oneparchoices}
	\question 软件测试是软件开发过程中重要和不可缺少的阶段,其包含的的内容和步骤甚多,而在测试过程的多种环节中最基本的是()\\
	\begin{oneparchoices}
		\choice 集成测试
		\correctchoice 单元测试
		\choice 系统测试
		\choice 验收测试
	\end{oneparchoices}
	\question 软件工程开发的可行性研究是决定软件项目是否继续开发的关键,而可行性研究的结论主要关于()\\
	\begin{oneparchoices}
		\correctchoice 软件系统目标
		\choice 软件的可测试性
		\choice 软件的功能
		\choice 软件的质量
	\end{oneparchoices}
	\question 结构化程序设计理论认为,实现良好的程序结构要应用()的分析方法。\\
	\begin{oneparchoices}
		\correctchoice 自顶向下
		\choice 自底向上
		\choice 面向对象
		\choice 基于组件
	\end{oneparchoices}
	\question 在下面列出的基本成分中,哪个不是数据流程图的基本成分?()\\
	\begin{oneparchoices}
		\choice 信息处理
		\choice 信息储存
		\choice 外部实体
		\correctchoice 系统状态
	\end{oneparchoices}
	\question PAD(\uline{Proble} \uline{Iiinlysis} \uline{Diagam})图是一种()工具。\\
	\begin{oneparchoices}
		\choice 系统描述
		\choice 详细设计
		\choice 测试
		\correctchoice 编程辅助
	\end{oneparchoices}
	\question 程序设计属于软件开发过程的()阶段。\\
	\begin{oneparchoices}
		\correctchoice 设计
		\choice 编程
		\choice 实现
		\choice 编码
	\end{oneparchoices}
	\question 程序的三种基本控制结构,它们的共同点是()\\
	\begin{oneparchoices}
		\choice 不能嵌套使用
		\choice 只能用来写简单的程序\\
		\choice 已经用硬件实现
		\correctchoice 只有一个入口和一个出口
	\end{oneparchoices}
	\question 耦合是软件中各模块件互相联系的一种度量,耦合的强弱取决于模块间的复杂程度。耦合的若干种类中,耦合度最高的是()\\
	\begin{oneparchoices}
		\correctchoice 内容耦合
		\choice 非直接耦合
		\choice 数据耦合
		\choice 控制耦合
	\end{oneparchoices}
	\question 在软件工程中,软件测试的目的是()\\
	\begin{oneparchoices}
		\choice 实验性运行软件
		\correctchoice 发现软件错误\\
		\choice 证明软件是正确的
		\choice 找出软件中全部错误
	\end{oneparchoices}
	\question 下面哪一项不是软件设计规格说明中模块的内容?()\\
	\begin{oneparchoices}
		\choice 接口描述
		\choice 数据的组织
		\correctchoice 外部文件结构
		\choice 处理过程描述
	\end{oneparchoices}
	\question 需求分析的主要任务是()\\
	\begin{oneparchoices}
		\choice 确定软件系统的主要功能,即进行系统功能分析,提出软件系统的目标、范围与功能说明\\
		\choice 分析用户要求,将软件功能和性能描述为具体的规格说明书\\
		\choice 对问题定义阶段所确定的问题实现的可能性和必要性做出研究\\
		\correctchoice 建立软件系统的总体结构,子系统划分,并提出软件结构图
	\end{oneparchoices}
	\question 软件结构中,由一模块直接控制的其他模块数称为()\\
	\begin{oneparchoices}
		\choice 深度
		\choice 宽度
		\choice 扇入数
		\correctchoice 扇出数
	\end{oneparchoices}
	\question 在数据字典中,()给出了某个文件的定义,文件的定义通常也是列出其记录的组成。\\
	\begin{oneparchoices}
		\choice 加工
		\correctchoice 数据存储
		\choice 数据流
		\choice 数据项
	\end{oneparchoices}
	\question 面向数据流的软件设计方法,一般是把数据流图中的数据流划分为什么样的两种流,再将数据流图映射为软件结构?()\\
	\begin{oneparchoices}
		\choice 数据流与事务流
		\correctchoice 交换流与事务流
		\choice 信息流与控制流
		\choice 交换流与数据流
	\end{oneparchoices}
	\question 与早期的软件开发方式相比较,结构化周期法最重要的指导原则应该是()\\
	\begin{oneparchoices}
		\choice 自顶向下设计
		\correctchoice 分阶段开发
		\choice 逐步求精
		\choice 用户需求至少
	\end{oneparchoices}
	\question 软件计划是软件开发的早期和重要阶段,此阶段要求交互和配合的是()\\
	\begin{oneparchoices}
		\choice 设计人员和用户
		\correctchoice 分析人员和用户\\
		\choice 分析人员和设计人员
		\choice 编码人员和用户
	\end{oneparchoices}
	\question 数据字典是对数据定义信息的集合,它所定义的对象都包含于()\\
	\begin{oneparchoices}
		\correctchoice 数据流图
		\choice 程序框图
		\choice 软件结构
		\choice 方案图
	\end{oneparchoices}
	\question 在软件开发的各种资源中,()是最重要的资源。\\
	\begin{oneparchoices}
		\choice 开发工具
		\choice 方法
		\choice 硬件环境
		\correctchoice 人员
	\end{oneparchoices}
	\question 软件文档是软件工程实施中的重要成分,它不仅是软件开发各阶段的重要依据,而且也影响软件的()\\
	\begin{oneparchoices}
		\choice 可理解性
		\correctchoice 可维护性
		\choice 可扩展性
		\choice 可移植性
	\end{oneparchoices}
	\question 软件复审时,其主要的复审对象是()\\
	\begin{oneparchoices}
		\choice 软件结构
		\correctchoice 软件文档
		\choice 系统编程
		\choice 文档标准
	\end{oneparchoices}
	\question 判断树和判断表是用于描述结构化分析方法中()环节的工具。\\
	\begin{oneparchoices}
		\choice 功能说明
		\correctchoice 数据加工
		\choice 流程描述
		\choice 性能说明
	\end{oneparchoices}
	\question 在体系结构图这种概念模型中,矩形框表示()\\
	\begin{oneparchoices}
		\choice 处理过程
		\correctchoice 模块
		\choice 外部实体
		\choice 内部实体
	\end{oneparchoices}
	\question 主要用来描述系统状态及其转换方式的数据模式是()\\
	\begin{oneparchoices}
		\choice E-R图
		\choice 结构图
		\choice DFD图
		\correctchoice IPO图
	\end{oneparchoices}
	\question 单元测试的测试用例主要根据()的结果来设计。\\
	\begin{oneparchoices}
		\choice 需求分析
		\choice 源程序
		\choice 概要设计
		\correctchoice 详细设计
	\end{oneparchoices}
	\question 为了提高测试的效率,应该()\\
	\begin{oneparchoices}
		\choice 随机地选取测试数据\\
		\choice 取一切可能的输入数据作为测试数据\\
		\choice 在完成编码后制订软件的测试计划\\
		\correctchoice 选择发现错误可能性大的数据作为测试数据
	\end{oneparchoices}
	\question 软件维护是指()\\
	\begin{oneparchoices}
		\choice 维护软件的正确进行
		\choice 软件的配置更新\\
		\correctchoice 对软件的改进、适应和完善
		\choice 软件开发期的一个阶段
	\end{oneparchoices}
	\question 软件工程学的概念除指软件开发技术研究外,另一重要内容为()\\
	\begin{oneparchoices}
		\correctchoice 软件工程管理
		\choice 软件开发工具的培训\\
		\choice 开发人员培训
		\choice 软件工程环境
	\end{oneparchoices}
	\question 在软件工程中,当前用于保证软件质量的主要技术手段还是()\\
	\begin{oneparchoices}
		\choice 正确性证明
		\correctchoice 测试
		\choice 自动程序设计
		\choice 符号证明
	\end{oneparchoices}
	\question 软件测试计划开始与需求分析阶段,完成于()阶段。\\
	\begin{oneparchoices}
		\choice 需求分析
		\choice 软件设计
		\choice 软件实现
		\correctchoice 软件测试
	\end{oneparchoices}
	\question 下列哪一项不是软件危机的表现形式?()\\
	\begin{oneparchoices}
		\choice 软件需求定义不明确,易偏高用户需求\\
		\choice 软件生产高成本,价格昂贵\\
		\choice 软件的可维护性差\\
		\correctchoice 系统软件与应用软件的联系越来越困难
	\end{oneparchoices}
	\question 数据流图是描绘信息在软件系统中流动和处理情况的图形工具,下列哪一个图形符号代表数据存储?()\\
	\begin{oneparchoices}
		\choice 箭头
		\choice 圆框
		\choice 直线
		\correctchoice 开口方框
	\end{oneparchoices}
	\question 软件需求分析一般应确定的是用户对软件的()\\
	\begin{oneparchoices}
		\choice 功能需求
		\correctchoice 功能需求和非功能需求
		\choice 性能需求
		\choice 非功能需求
	\end{oneparchoices}
	\question 在软件生存周期的瀑布模型中一般包括计划、()、设计、编码、测试、维护等阶段。\\
	\begin{oneparchoices}
		\choice 可行性分析
		\choice 需求采集
		\choice 需求分析
		\correctchoice 问题定义
	\end{oneparchoices}
	\question 软件设计中,可应用于详细设计的工具有()\\
	\begin{oneparchoices}
		\correctchoice 程序流程图、PAD图、方框图和伪码\\
		\choice 数据流程图、PAD图、结构图和伪码\\
		\choice 业务流程图、N-S图和伪码\\
		\choice 数据流程图、PAD图、N-S图和伪码
	\end{oneparchoices}
	\question 软件需求分析阶段的测试手段一般采用()\\
	\begin{oneparchoices}
		\choice 总结
		\choice 阶段性报告
		\correctchoice 需求分析评审
		\choice 不测试
	\end{oneparchoices}
	\question 程序流程图是一种传统的程序设计表示工具,有其优点和缺点,使用该工具时应注意()\\
	\begin{oneparchoices}
		\choice 支持逐步求精
		\choice 考虑控制流程
		\correctchoice 遵守结构化设计原则
		\choice 数据结构表示
	\end{oneparchoices}
	\question 从软件的开发到运行的全过程,软件文档的重要作用是众所周知的,但执行时差距甚大,其根本原因是()\\
	\begin{oneparchoices}
		\choice 文档规范程度低
		\choice 文档生成工具差
		\choice 开发者缺乏重视
		\correctchoice 工程化程度尚低
	\end{oneparchoices}
	\question 软件测试是保证软件质量的重要措施,它的实施应该是在()\\
	\begin{oneparchoices}
		\choice 程序编码阶段
		\choice 软件设计阶段
		\correctchoice 软件开发全过程
		\choice 软件运行阶段
	\end{oneparchoices}
	\question 在软件的结构化设计(SD)方法中,一般分为总体设计和详细设计两个阶段,其中总体设计主要是建立()\\
	\begin{oneparchoices}
		\correctchoice 软件结构
		\choice 软件流程
		\choice 软件模型
		\choice 软件模块
	\end{oneparchoices}
	\question 瀑布模型把软件生存周期划分为软件定义、软件开发与()三个阶段,而每个阶段又可分为若干更小的阶段。\\
	\begin{oneparchoices}
		\choice 详细设计阶段
		\choice 可行性研究阶段
		\correctchoice 运行及维护
		\choice 问题定义
	\end{oneparchoices}
	\question 软件工程的结构化生命周期方法中将软件生命周期分为若干阶段,软件详细设计是属于()阶段\\
	\begin{oneparchoices}
		\choice 计划阶段
		\correctchoice 开发阶段
		\choice 运行阶段
		\choice 维护阶段
	\end{oneparchoices}
	\question 最适合于记录各种细节的概念模式是()\\
	\begin{oneparchoices}
		\choice 实体关系图
		\correctchoice 数据字典
		\choice 结构图
		\choice 框图
	\end{oneparchoices}
	\question 在软件的分析阶段,常用()来描述业务处理系统的信息来源、储存、处理和去向\\
	\begin{oneparchoices}
		\choice E-R图
		\choice 框图
		\correctchoice DFD
		\choice 时序网络
	\end{oneparchoices}
	\question 与设计测试数据无关的文档是()\\
	\begin{oneparchoices}
		\choice 需求说明书
		\choice 设计说明书
		\choice 源程序
		\correctchoice 项目开发计划
	\end{oneparchoices}
	\question 模块的内聚是从功能的角度来度量模块内的联系,内聚度最强的是()\\
	\begin{oneparchoices}
		\choice 通信内聚
		\correctchoice 功能内聚
		\choice 顺序内聚
		\choice 逻辑内聚
	\end{oneparchoices}
	\question 结构化分析方法以数据流图、()和加工说明等描述工具,即用直观的图和简洁的语言来描述软件系统模型。\\
	\begin{oneparchoices}
		\choice DFD图
		\choice PAD图
		\choice IPO图
		\correctchoice 数据字典
	\end{oneparchoices}
	\question 在软件工程中,高质量的文档标准是完整性、一致性和()\\
	\begin{oneparchoices}
		\choice 统一性
		\choice 完全性
		\correctchoice 无二义性
		\choice 组合性
	\end{oneparchoices}
	\question 结构图中,带有注释的小箭头表示()\\
	\begin{oneparchoices}
		\choice 模块
		\correctchoice 调用
		\choice 数据
		\choice 模块间判断
	\end{oneparchoices}
	\question 下列叙述中不是关于有利于软件可维护性的描述是()\\
	\begin{oneparchoices}
		\choice 在进行需求分析时应考虑维护问题\\
		\choice 使用维护工具和支撑环境\\
		\correctchoice 在进行总体设计时,应加强模块之间的联系\\
		\choice 重视程序结构的设计,使程序具有较好的层次结构
	\end{oneparchoices}
	\question 在软件质量因素中,软件在异常条件下仍能运行的能力称为软件的()\\
	\begin{oneparchoices}
		\choice 可用性
		\correctchoice 健壮性
		\choice 可靠性
		\choice 安全性
	\end{oneparchoices}
	\question 软件设计包括总体设计和详细设计两部分,下列陈述中哪个是详细设计的内容?()\\
	\begin{oneparchoices}
		\choice 软件结构
		\correctchoice 模块算法
		\choice 制定测试计划
		\choice 数据库设计
	\end{oneparchoices}
	\question 由事务型数据流图映射为软件结构的设计首先应设计一个(),它有两个功能,接受事务数据,另一个是根据事务类型调度相应的处理模块。\\
	\begin{oneparchoices}
		\choice 总控模块
		\correctchoice 事务中心
		\choice 交换中心
		\choice 接收分支
	\end{oneparchoices}
	\question 软件测试是为了()而执行程序的过程。\\
	\begin{oneparchoices}
		\choice 纠正错误
		\correctchoice 发现错误
		\choice 避免错误
		\choice 证明正确
	\end{oneparchoices}
	\question 在瀑布模型中,将软件开发划分为若干个时期,软件项目的可行性研究一般被归属于()\\
	\begin{oneparchoices}
		\choice 维护时期
		\correctchoice 定义时期
		\choice 运行时期
		\choice 开发时期
	\end{oneparchoices}
	\question 软件设计中,设计复审是和计划本身一样重要的环节,其主要目的和作用是()\\
	\begin{oneparchoices}
		\choice 减少测试工作量
		\choice 避免后期付出高代价\\
		\correctchoice 保证软件质量
		\choice 缩短软件开发周期
	\end{oneparchoices}
	\question 软件危机通常是指在计算机软件开发和维护中所产生的一系列严重问题,这些问题中相对次要的因素是()\\
	\begin{oneparchoices}
		\choice 软件功能
		\choice 文档质量
		\choice 开发效率
		\correctchoice 软件性能
	\end{oneparchoices}
	\question 软件工程的结构化生命方法是将软件开发的全过程划分为相互独立而又相互依存的阶段,软件的逻辑模型形成于()\\
	\begin{oneparchoices}
		\choice 开发阶段
		\choice 计划阶段
		\correctchoice 分析阶段
		\choice 设计阶段
	\end{oneparchoices}
	\question 软件工程的基本要素包括方法、工具和()\\
	\begin{oneparchoices}
		\correctchoice 过程
		\choice 软件系统
		\choice 硬件环境
		\choice 人员
	\end{oneparchoices}
	\question 模块内聚度越高,说明模块内容各成分彼此结合的程度越()\\
	\begin{oneparchoices}
		\choice 松散
		\correctchoice 紧密
		\choice 无法判断
		\choice 相等
	\end{oneparchoices}
	\question 软件生产的成败更多的依赖于()\\
	\begin{oneparchoices}
		\choice 领导者的指挥才能
		\choice 程序员个人的编程能力\\
		\correctchoice 合理地组织与协调
		\choice 用户的配合
	\end{oneparchoices}
	\question 模块本身的内聚是模块独立性的重要度量因素之一。在七类内聚中,具有最强内聚的一类是()\\
	\begin{oneparchoices}
		\choice 顺序性内聚
		\choice 过程性内聚
		\choice 逻辑性内聚
		\correctchoice 功能性内聚
	\end{oneparchoices}
	\question 软件详细设计采用的方法是()\\
	\begin{oneparchoices}
		\choice 结构化程序设计
		\choice 模块设计
		\correctchoice 结构化设计
		\choice PDL语言
	\end{oneparchoices}
	\question ()在开发软件时,可用来提高程序员的工作效率。\\
	\begin{oneparchoices}
		\correctchoice 程序开发环境
		\choice 操作系统的作业管理功能\\
		\choice 编译程序的优化功能
		\choice 并行运算的大型计算机
	\end{oneparchoices}
	\question 结构设计是一种应用最广泛的系统设计方法,是以()为基础、自顶向下、逐步求精和模块化的过程。\\
	\begin{oneparchoices}
		\choice 数据流
		\correctchoice 数据流图
		\choice 数据库
		\choice 数据结构
	\end{oneparchoices}
	\question 可行性研究后得出的结论主要与()有关。\\
	\begin{oneparchoices}
		\correctchoice 软件系统目标
		\choice 软件的功能
		\choice 软件的性能
		\choice 软件的质量
	\end{oneparchoices}
	\question 软件设计阶段的输出主要是()\\
	\begin{oneparchoices}
		\choice 程序
		\choice 模块
		\choice 伪代码
		\correctchoice 设计规格说明书
	\end{oneparchoices}
	\question 使用表示结构化控制结构的问题分析图(PAD)符号所设计出来的程序()\\
	\begin{oneparchoices}
		\correctchoice 必然是结构化程序
		\choice 一般不是结构化程序
		\choice 一般是结构化程序
		\choice 绝对不是结构化程序
	\end{oneparchoices}
	\question 软件测试方法中,黑盒、白盒测试法是常用的方法,其中白盒测试主要用于测试()\\
	\begin{oneparchoices}
		\choice 结构合理性
		\choice 软件外部功能
		\choice 程序正确性
		\correctchoice 程序内部逻辑
	\end{oneparchoices}
	\question 数据流图中,下列哪一种数据流的流向是不可能发生的?()\\
	\begin{oneparchoices}
		\choice 从加工流向加工
		\correctchoice 从数据存储流向外部实体\\
		\choice 从加工流向外部实体
		\choice 从外部实体流向加工
	\end{oneparchoices}
	\question 确定每一个模块使用的数据结构属于软件设计的哪一个阶段?()\\
	\begin{oneparchoices}
		\choice 总体设计
		\choice 需求分析
		\choice 编程调试
		\correctchoice 详细设计
	\end{oneparchoices}
	\question 数据字典是用来定义()中的各个成分的具体含义的。\\
	\begin{oneparchoices}
		\choice 流程图
		\choice 功能结构图
		\choice 系统结构图
		\correctchoice 数据流图
	\end{oneparchoices}
	\question 软件的复杂性是(),它引起人员通信困难、开发费用超支、开发时间超时等问题。\\
	\begin{oneparchoices}
		\correctchoice 固有的
		\choice 人为的
		\choice 可消除的
		\choice 不可降低的
	\end{oneparchoices}
	\question 在软件生存周期的瀑布模型中一般包括计划。()、设计、编码、测试、维护等阶段。\\
	\begin{oneparchoices}
		\choice 可行性分析
		\choice 需求采集
		\choice 需求分析
		\correctchoice 问题定义
	\end{oneparchoices}
\end{questions}
\section{简答题}
\begin{enumerate}
	\item 如何理解模块独立性?衡量模块独立性的两个标准是什么?它们各表示什么含义?
	\item 什么是黑盒测试?黑盒测试的方法有哪些?
	\item 什么是软件维护?软件维护有哪几种类型?
	\item 什么是需求分析?需求分析阶段的基本任务是什么?
	\item 为什么软件需要维护?简述软件维护的过程。
\end{enumerate}
\end{document}